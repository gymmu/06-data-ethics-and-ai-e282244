\documentclass{article}

\usepackage[ngerman]{babel}
\usepackage[utf8]{inputenc}
\usepackage[T1]{fontenc}
\usepackage{hyperref}
\usepackage{csquotes}

\usepackage[
    backend=biber,
    style=apa,
    sortlocale=de_DE,
    natbib=true,
    url=false,
    doi=false,
    sortcites=true,
    sorting=nyt,
    isbn=false,
    hyperref=true,
    backref=false,
    giveninits=false,
    eprint=false]{biblatex}
\addbibresource{../references/bibliography.bib}

\title{Review des Papers "Ist der Einsatz von KI im Gesundheitswesen ethisch vertretbar?" von Käthe Aguirre Linke}
\author{Lars Burkard}
\date{\today}

\parindent=0em

\begin{document}
\maketitle

\abstract{
    Das Paper von Käthe Linke Aguirre untersucht, ob der Einsatz von Künstlicher Intelligenz im Gesundheitswesen ethisch vertretbar ist. Es werden dabei verschiedene Anwendungen von KI in der Medizin geschildert, dazu gehören Diagnostik, Prognose, Behandlungsplanung und Unterstützung sowie die damit verbundenen Richtlinien und sozialen Herausforderungen.
}
\vspace{15mm}\tableofcontents

\printbibliography


\vspace{117mm}

\section{Positive Askpete}
\vspace{3mm}\subsection{KI im Gesudheitswesen}
 Da das Dokument einen umfassenden Überblick über die Anwendung von KI im Gesundheitswesen bietet, von Diagnostik bis hin zu Behandlungsunterstützung, hilft es dem Leser, die Auswirkungen von KI in der Medizin zu verstehen.

 \vspace{2mm}Das Dokument ist klar strukturiert, dabei hast du gute Abschnitte und Titel genutzt, sodass die einzelnen Themen schnell verständlich sind.

 \vspace{2mm}Da du gute Argumente und ethische Fragen, wie Datenschutz, Transparenz und Sicherheit verwendet hast, entsteht ein Aufbau von Vertrauen in die KI-Technologie im Gesundheitswesen und macht uns als potenzielle Patienten etwas beruhigter in Hinsicht der KI-Nutzung während unseres Heilungsprozesses.

 \vspace{2mm}Das Einbeziehen von guten Quellen übermittelt eine gewisse Glaubwürdigkeit gegenüber deinen dargestellten Argumenten und zeigt auch, dass du dich gut informiert hast und dich auch für das Thema interessierst.

\vspace{3mm}\subsection{Was ist KI?} 
Es wird klar und verständlich beschrieben, was KI überhaupt ist, und ist besonders für Leser ohne Vorkenntnisse sehr hilfreich.

\vspace{2mm}Da du uns auch klare Beispiele von heutigen Einsatzbereichen schilderst, macht es dein Konzept verständlich und zeigt auch die reale Bedeutung von Anwendungsmöglichkeiten.


\vspace{8mm}\subsection{Wie wird KI Trainiert?} 
 Es gibt eine Strukturierung der Trainingsabschnitte (Training, Validierung, Testen), es ermöglicht dem Leser, einen groben Überblick zu verschaffen, wie die eigenen „Assistenten“ überhaupt geschult werden. Die Unterteilung einzelner Schritte mit Absätzen trägt zusätzlich zum Verständnis des Themas bei.

 \vspace{2mm}Du gehst auch kritisch an das Thema heran, indem du die Beachtung von Datenschutzrichtlinien und ethischen Überlegungen stark in den Vordergrund rückst. Du zeigst dabei klar, welche sozialen und rechtlichen Rahmenbedingungen dabei zu beachten sind.

\section {Bereiche zur Verbesserung}
Obwohl du im Dokument eine breite Palette von Themen abdeckst, könntest du das nächste Mal bei einigen Abschnitten vertieftere Analysen einbeziehen. Beispielsweise durch Studien, die die ethischen Punkte untermauern und das Ganze noch vertrauenswürdiger machen.

\vspace{2mm}Zusätzlich könnten auch Zitate oder Anekdoten eingebaut werden, wodurch durch Experten mehr Klarheit verschafft wird.

\vspace{2mm}Auch eine Veranschaulichung, wie zum Beispiel bei den Trainingsabschnitten, könnte durch Diagramme oder Ähnliches verdeutlicht werden.

\vspace{2mm}Leider gibt es auch vermehrt Rechtschreibfehler, wodurch dem Leser die Glaubwürdigkeit deines Themas etwas genommen wird, da es einen unprofessionellen Eindruck hinterlässt. Lass doch dein Dokument in Zukunft nochmals von einer anderen Person überprüfen oder lasse es von einer Künstlichen Intelligenz korrigieren.



\vspace{10mm}\section{Fazit} 

Insgesamt gesehen ist dein Dokument sehr gut strukturiert und behandelt ein wichtiges Thema, das für uns alle an Bedeutung gewinnt. Durch die Behebung meiner genannten Verbesserungsbereiche kann das Ganze zu einer noch wertvolleren Ressource für das Verständnis der ethischen Einbeziehung in unserem Gesundheitswesen werden.

\end{document}
