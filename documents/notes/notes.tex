\documentclass{article}

\usepackage[ngerman]{babel}
\usepackage[utf8]{inputenc}
\usepackage[T1]{fontenc}
\usepackage{hyperref}
\usepackage{csquotes}

\usepackage[
    backend=biber,
    style=apa,
    sortlocale=de_DE,
    natbib=true,
    url=false,
    doi=false,
    sortcites=true,
    sorting=nyt,
    isbn=false,
    hyperref=true,
    backref=false,
    giveninits=false,
    eprint=false]{biblatex}
\addbibresource{../references/bibliography.bib}

\title{Notizen zum Projekt Data Ethics}
\author{Lars Burkard}
\date{\today}

\parindent=0em


\begin{document}
\maketitle

\abstract{
    Dieses Dokument ist eine Sammlung von Notizen zu dem Projekt. Die Struktur innerhalb des
    Projektes ist gleich ausgelegt wie in der Hauptarbeit, somit kann hier einfach geschrieben
    werden, und die Teile die man verwenden möchte, kann man direkt in die Hauptdatei ziehen.
}

\tableofcontents

\section{Fragestellung
    \label{sec:fragestellung}}


    Welche ethischen Herausforderungen entstehen durch den Einsatz von KI im Gesundheitswesen?

\section{KI in der Gesunheit}
In der Arbeit werden die ethischen Schwierigkeiten und Einschränkungen der künstlichen Intelligenz im Gesundheitswesen dargestellt. Im ethischen Rahmen werden sowohl die Potenziale als auch die Einschränkungen dieser Instrumente behandelt, und dies wird anhand von Anwendungsbeispielen illustriert.

\vspace{2mm}Die Künstliche Intelligenz hat vielfältige Anwendungsmöglichkeiten im Bereich der Gesundheitsversorgung, darunter Diagnose, Therapie, Pflege, Verwaltung und Forschung. Es ist besonders wichtig, die Vertrauenswürdigkeit zu gewährleisten, da ethische Rahmenbedingungen zu beachten sind.

\vspace{27mm}Im Gesundheitswesen kommt KI zum Einsatz, um eine individuellere und verbesserte medizinische Versorgung zu schaffen. Es erlaubt eine enge Kooperation von Mensch und Maschine zur Einschätzung von Krankheitsrisiken sowie zur Beeinflussung und Veränderung verschiedener Bereiche des Gesundheitswesens. Die Einsatzmöglichkeiten erstrecken sich von elektronischen Patientenakten bis zu Robotern zur Operation und Therapie. Die KI bringt insgesamt zahlreiche Vorteile für die Medizin mit sich.

\vspace{2mm}Medikamenteneinnahmezyklen, die Einhaltung jeweiliger Ernährungsempfehlungen und die daraus resultierenden Veränderungen im Lebensstil sind Beispiele für die Folgeleistungen medizinischer Anwendungen. 


\vspace{2mm}Die Anwendung von KI kann dazu führen, dass die Selbstbestimmung der Nutzer eingeschränkt wird, unangemessenes Vertrauen geschaffen wird und Manipulationen auftreten, die normalerweise nicht vorkommen würden. Es besteht die Möglichkeit, dass Risiken entstehen, die Widerstand und Kritik auslösen und somit zu einer Skepsis in der gesamten Bevölkerung führen. Die Anwendung von KI stellt ethische Fragen dar und kann in Unternehmen und bei Mitarbeitern zu Spannungen führen.



\section{Was ist KI?}

\vspace{2mm}\textbf{Definition:}

\vspace{2mm}KI, geprägt von John McCarthy, nachbildet menschliche Intelligenz

\vspace{2mm}\textbf{Komponenten:}

\vspace{2mm}Sensoren (Datenerfassung), Informationsverarbeitung (Entscheidungsfindung), Aktoren (Handlungen)
Relevanz: Unverzichtbar in der Gesellschaft, unterstützt bei komplexen Problemen

\vspace{2mm}\textbf{Arten von KI:}

\vspace{2mm}Schwache KI: Einfache Aufgaben, präzise Ausführung
Starke KI: Menschenähnliche Intelligenz, erfordert Verständnis menschlicher Geistesfunktionen, ethische Herausforderungen
Einsatzbereiche:

\vspace{2mm}\textbf{Gesundheitswesen:}

\vspace{2mm}Präzisere Diagnosen, effizientere Behandlungen
COVID-19-Pandemie: Erhöhte Bedeutung von KI
Dokumentinhalt: Schwierigkeiten und ethische Richtlinien beim Einsatz von KI im Gesundheitswesen

\vspace{23mm}

\section{Training von KI}
\textbf{Potenzial der KI:}

\vspace{2mm}Selbstfahrende Autos und Roboter

\vspace{2mm}\textbf{Trainingsprozess drei Teilschritte:}

Training
Validierung
Testen

\vspace{2mm}\textbf{Trainingsprozess im Detail:}

Algorithmen werden mit Daten gefüttert
Vorhersagen und Genauigkeit verbessern
Maschinelles Lernen: kontinuierliche Verbesserung durch Daten
Deep Learning: tiefe neuronale Netze erkennen Muster
Arten des Trainings:

\vspace{2mm}Überwachtes Lernen: gelabelte Daten, kontinuierliches Training

Unüberwachtes Lernen: Mustererkennung ohne gelabelte Daten

Validierung: Prüfung der Leistung mit neuen Daten, Entscheidung über Modellverbesserung oder Abschluss des Trainings

\vspace{2mm}Testen: Prüfung der gelernten KI in der realen Welt




\printbibliography

\end{document}
